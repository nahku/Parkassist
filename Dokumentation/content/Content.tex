%!TEX root = ../dokumentation.tex

\chapter{D1: Aufwandsabschätzung nach der Dreipunktmethode}\label{cha:D1}
\begin{table}[H]
\centering
\caption{Dreipunktabschätzung des Aufwands der Anforderungen}
\begin{adjustbox}{width=1\textwidth, center=\textwidth}
\renewcommand{\arraystretch}{1}
\begin{tabular}{lllllll}
\textbf{Anforderung} \textbf{Optimistisch} & \textbf{Wahrscheinlich} & \textbf{Pessimistisch} & \textbf{<T>} & \textbf{sigmahoch2} & \textbf{wirklich}\\\hline
D1 & .& .& .& .& .&\\
\end{tabular}

\end{adjustbox}
\label{tbl:ConceptTPTPProductionSymbols}
\end{table}
\chapter{D2: Machbarkeitsdemonstration}\label{cha:D2}

\chapter{D3: Anlyse des menschlichen Geschwindigkeitsprofils}\label{cha:D3}

\chapter{D4*: Betrachtung von Unebenheiten des Parkplatzes}\label{cha:D4}

\chapter{D5: Betrachtung von Unsicherheiten in der Geschwindigkeitsmessung}\label{cha:D5}
validate findings by numbers from simulation

\chapter{D6: Implementierung des Pulssignals in Simulink}\label{cha:D6}

\chapter{D7: Übernahme des Simulinkmodells nach ASCET}\label{cha:D7}

\chapter{D8: Implementierung des Pulssignals in ASCET}\label{cha:D8}

\chapter{D9: Unit-Tests für das Pulssignal in ASCET}\label{cha:D9}

\chapter{D10: Entwicklung und Druchführung von Systemtests für die ASCET Simulation}\label{cha:D10}

\chapter{D11*: Plausibilitätsprüfung gemessener Geschwindigkeiten und  Strecken gegeneinander}\label{cha:D11}

\chapter{D13*: Einfluss von Ungenauigkeiten}\label{cha:D13}

\chapter{D14*: Reflexion}\label{cha:D14}